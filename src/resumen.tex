\section{Resumen número $\Pi$}
\label{1:sec:1}

El primer matemático que empleó $\Pi$ como símbolo del cociente entre el perímetro de un círculo y la longitud de su diámetro fue Leonhard Eüler, en 1737, siendo admitido como símbolo estándar.

$\Pi$ (pi) es la relación entre la longitud de una circunferencia y su diámetro, en geometría euclidiana. Es un número irracional y una de las constantes matemáticas más importantes. Se emplea frecuentemente en matemáticas, física e ingeniería. El valor numérico de $\Pi$, truncado a sus primeras cifras, es el siguiente:

   3,14159265358979323846

El valor de $\Pi$ se ha obtenido con diversas aproximaciones a lo largo de la historia, siendo una de las constantes matemáticas que más aparece en las ecuaciones de la física, junto con el número e. Cabe destacar que el cociente entre la longitud de cualquier circunferencia y la de su diámetro no es constante en geometrías no euclídeas.